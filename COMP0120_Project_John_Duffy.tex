\documentclass[10pt, a4paper]{amsart}

\usepackage{amsmath}
\usepackage{graphicx}
\usepackage{float}
\usepackage{caption}
\usepackage{subcaption}

\title{COMP0120 Project - Support Vector Machines (SVM\MakeLowercase{s})}
\author{\textbf{John Duffy, Student Number: 19154676}}

\begin{document}

\maketitle

\section{Introduction}

Suppose we have been provided with a dataset $L$ consisting of $N$ data points, each with a $x_0$ and $x_1$ value. Furthermore, suppose we have been provided with a vector $y$ of length $N$ consisting of the values $a$ and $b$, where each index of $y$ corresponds to the same index of $L$. So, each item in the dataset $L$ has an associated value $a$ or $b$. We could plot the data points of $L$ as per Figure 1. Clearly, the data points with the value $a$ are clustered together, as are those with the value $b$, they are $linearly$  $separable$.

Now suppose we wish to draw a line $l$ between the two clusters, such that additional data points without a known value can be accurately classified as being of value $a$ or $b$. This process is called \emph{linear binary classification}; \emph{linear} because we wish to separate the classes (the clusters of values $a$ and $b$) with a line, and \emph{binary} because we have two classes. This process can be extended to be $non-linear$, i.e. use a higher degree polynomial to separate the classes, and to include additional classes.  

The classification process outlined above, using a method called Support Vector Machines (SVMs), is the subject of this report.

\begin{figure}
	\centering	
	\includegraphics[width=1.0\textwidth]{intro_0.pdf}
	\caption{Dataset $L$}
\end{figure}

How should we decide where to draw the line $l$? Consider Figure 2. If we used the line $l_1$ then additional data points of Class $a$ may be classified incorrectly as Class $b$. Conversely, if we used the line $l_2$ then additional data points of Class $b$ may be incorrectly classified as Class $a$.

\begin{figure}
	\centering	
	\includegraphics[width=1.0\textwidth]{intro_1.pdf}
	\caption{Dataset $L$}
\end{figure}

SVMs, first conceived of by Cortes and Vapnik [1], provide a method of deciding how to draw the line $l$ (and higher degree polynomials in the non-linear case). SVMs maximise the width $W$ of the "street" between classes, as depicted in Figure 3.

\begin{figure}
	\centering	
	\includegraphics[width=1.0\textwidth]{intro_2.pdf}
	\caption{SVM linear separation of Class $a$ and Class $b$ by maximising the width $W$ of the "street" between the classes.}
\end{figure}




\section{Mathematical Setting - Summary}

\subsection{Binary Classification}\hfill

Primal Problem


%\begin{figure}
%	\centering	
%
%	\begin{subfigure}{0.5\textwidth}
%		\centering
%		\includegraphics[width=1.0\textwidth]{iris_sepal.eps}
%		\caption{Image 1}
%	\end{subfigure}%
%	\begin{subfigure}{0.5\textwidth}
%		\centering
%		\includegraphics[width=1.0\textwidth]{iris_petal.eps}
%		\caption{image 2}
%	\end{subfigure}
%	\caption{Iris Dataset Petal Data}
%\end{figure}

Dual Problem

\subsection{Hard Margin}

\subsection{Soft Margin}

\subsection{Challenge!}\hfill

More classes?

Different Penalty Functions?

Algorithms not on syllabus?

\subsection{Non-Linear Classification}\hfill

Kernels?


\section{Simulation Study}

\subsection{Chosen Data Set}\hfill

\begin{figure}
	\centering	
	\includegraphics[width=1.0\textwidth]{iris_sepal.pdf}
	\caption{Iris Dataset Sepal Data}
\end{figure}

\begin{figure}
	\centering	
	\includegraphics[width=1.0\textwidth]{iris_petal.pdf}
	\caption{Iris Dataset Petal Data}
\end{figure}

Challenges?

How to address?

\subsection{Resulting Optimisation Problem}\hfill

Convex/Non-Convex

Constrained/Unconstrained

Smooth/Non-Smooth

Linear/Quadratic/Non-Linear

Challenges


\section{Solution of Optimisation Problem}

\subsection{Algorithm 1}\hfill

\subsection{Algorithm 2}\hfill


\section{References}

[1] ???

[2] ???

[3] ???

[4] Scikit-learn: Machine Learning in Python, Pedregosa et al., JMLR 12, pp. 2825-2830, 2011.

\end{document}
